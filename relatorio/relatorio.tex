\documentclass[11pt,a4paper]{report}
\usepackage[a4paper,left=3cm,right=2cm,top=2.5cm,bottom=2.5cm]{geometry}
\usepackage[colorlinks=true,linkcolor=blue,citecolor=blue]{hyperref}
\usepackage[T1]{fontenc} % to have good hyphenation
\usepackage[utf8]{inputenc} % accented characters in input
\usepackage[portuguese]{babel}
\usepackage{color}
\usepackage{adjustbox}
\usepackage{listings}
\lstset{language=C}
\begin{document}
\title{Trabalho Prático nº1 \\FLex}
\author{
   Sérgio Oliveira~\\
   \texttt{a62134}
   }
\date{10 de Outubro de 2019}
\maketitle
\raggedbottom
\pagebreak
\pagebreak

% CAP 0 - - - - - - - -  - - - - 

\tableofcontents
\pagebreak
\chapter{Introdução}

Este relatório tem como objetivo mostrar o desenvolvimento de um programa escrito em formato Lex, que faz o processamento de linhas de um ficheiro passado para input, filtrando os conteúdos para as variáveis corretas.
\\
Dos exemplos juntos em anexo com o enunciado do trabalho prático, fazemos uso do ficheiro em formato \textit{XML} \textbf{exemplo-Enamex} para o 1º capítulo do relatório. Um exemplo em formato \textit{TXT} não anexado com o enunciado é usado no processamento de tags personalizadas e resumidas para gerar marcas conhecidas de \textbf{HTML} para criar um documento ...

\raggedbottom
\pagebreak

% CAP 1 - - - - - - - -  - - - - 

\chapter{Enamex}
\section{Nomes}

A funcionalidade básica deste trabalho está em identificar padrões da linguagem \textit{Enamex} para processar palavras-chave, assim como nomes completos de pessoas, locais e cidades.

Depois de processado, é então gerada uma página HTML com toda a informação extraída do exemplo \textit{Enamex}, filtrando a informação que nos é pedida.
~\\
Como no exemplo em anexo junto com o trabalho prático, existem diversos caractéres especiais vindos do tipo de linguagem \textit{XML}, portanto primeiro temos de fazer o processamento dos mesmos.



\pagebreak
\section{Países e cidades}
\pagebreak
\section{Ordenação}


\raggedbottom
\pagebreak
% CAP 2 - - - - - - - -  - - - - 

\chapter{WIKI}
\section{De FLex para HTML}

\pagebreak
\section{Problemas}


% CAP 3 - - - - - - - -  - - - - 


\chapter{Conclusão}

Após dias de desenvolvimento, há que realçar as principais dificuldades. O facto de não poder usar os conjuntos de letras, dígitos e caractéres especiais (nomeadamente \textbackslash d, \textbackslash s e \textbackslash w), devido a prováveis limitações de libraries e visto que dava problemas na compilação em FLex e gcc, foi posta de parte a introdução desses caracteres.

Após várias correcções e testes ao trabalho, corrigiu-se esse precalço.

Com o desenvolvimento destes projetos, haverá também sempre a expansão do nosso conhecimento e interesse em sistemas UNIX e UNIX-like, assim como em Expressões Regulares.

\end{document}